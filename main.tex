\documentclass{article}
\usepackage[french]{babel}
\usepackage[a4paper,top=2cm,bottom=2cm,left=3cm,right=3cm,marginparwidth=1.75cm]{geometry}
\usepackage{amsmath}
\usepackage[T1]{fontenc}
\usepackage{graphicx}
\usepackage[colorlinks=true, allcolors=blue]{hyperref}

\title{L'implication éthique liée à la création de logiciels}
\author{Lynne Guizani, Sephora Mavitidi Bunga, Andrea Lafarge, Bercy Yola}

\begin{document}
\maketitle

\section{Introduction}

\subsection{Lynne}

\subsection{Andrea}


\subsection{Sephora}


\subsection{Les lois mise en rigueurs et les dangers d'un manque d'étique dans la création de logiciel (Yola)}

Un problème d'éthique dans le code d'un logiciel peut entraîner le non-respect des lois.\\
\\Par exemple l'IA COMPAS utilisé un temps dans les tribunaux Américain n'aurai jamais pu être mit en place en France car à la suite de recherches, nous avons appris que cette IA portait atteinte à l'identité de certaines personnes ce qui est non conforme au premier article de la loi relative à l'informatique, aux fichiers et aux libertés dont je parlerai pendant l'entièreté de cette partie.\\
\\En effet cette IA qui devait calculer le taux de possible récidive des condamnés semblait être entraîner à donner un score de récidive plus élevé aux personnes noires qu’aux personnes blanches malgré que les mêmes types de profiles soient identifiés pour les deux groupes.\\
\\
(Graphique)
\\
\\Il en va de même avec les programmes de reconnaissance faciales et d'identification de l'orientation sexuelle qui peut si coder sans conformité à l'éthique porter atteinte à certains groupes de personnes.\\
\\Ce sont des choses déjà vu en cours mais qui je pense ont totalement leur place dans ce sujet.\\
\\Produire des logiciels ne possédant pas un code éthique est aussi problématique que dangereux pour les développeurs et les utilisateurs.\\
\\Tout d'abord le manque de protection peut mettre en dangers les informations personnels des utilisateurs et les laisser fuiter entre les mains des personnes mal intentionnés c'est ici que la cybersécurité joue une place importante pour assurer la conformité d'un code pour éviter ce genre d'événements de se produire dans le même cas les utilisateurs ne feront plus confiances à ces logiciels qui mettra donc en péril le travail des développeurs et la réputation de l'entreprise derrière.\\
\\Ensuite un code non éthique peut-être défavorable pour certaines personnes certain algorithmes coder sans contrainte éthiques finissent par être problématique et peuvent entraîner des discriminations pour les personnes handicapées par exemple, ce que Sephora mettra en lumière dans sa partie mais aussi des discriminations raciales qui vont donc à l'encontre des lois françaises comme j'ai pu le dire plus tôt avec l'exemple de l'IA COMPAS aux États-Unis.\\
\\Enfin le danger est aussi pour la planète, un code non éthique et non optimisé entraînera un certain impact sur l'environnement ce qui sera étudié en profondeur par Lynne.\\


\subsection{Conclusion}

\subsection{Retour Réflexif}


\bibliographystyle{alpha}
\bibliography{sample}

\end{document}